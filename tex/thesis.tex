\documentclass{article}
\usepackage{graphicx} % Required for inserting images

\title{NanoNeat}
\author{Przemysław Pilipczuk}
\date{December 2023}

\begin{document}

\maketitle

\tableofcontents
\section{Introduction}
\subsection{Abstract}
In this work I have implemented a fully functional NeuroEvolution of Augumenting Topologies (NEAT) algorithm 
with a focus on simplicity and readibility. The core algorithm is ~400loc of Python, and therefore is a 
good resource for learning about NEAT. 
\subsection{Keywords}
Neuroevolution, NEAT, Evolutionary Algorithms, Machine Learning, Artificial Intelligence, Didactic Implementation
\section{Related work}
todo[idk, write about original paper here? or maybe about other implemetations?]
\section{Methodology}
    \subsection{Problems used to test the implementation}
        \subsubsection{XOR}
        Xor problem is a simple problem used mostly to test if the algorithm can start generating structure. 
        This is because XOR function is non linearly separable - it cannot be solved by the simple topology that neat starts with.
        To solve it, the algorithm needs to evolve a hidden layer. 
        This problem is also used to test if the algorithm can generate minimal structure - the original paper 
        todo[CITATION] states that the algoritm can solve the problem with 1 node in the hidden layer (However,
        it does require that the starting topology has an additional input node designated
        as bias, connected straight to the output). This requirement is not presented as a builtin configuration variant for this specific problem , as
        it is trivial to add it using basic configuration change (add 1 more input that always returns "1", and add a function that 
        hooks start of the population to set its connection weight to the output as 1).
        \subsubsection{Cartpole}
        Cartpole is a problem taken from Farama Foundation's Gymnasium library. It is a classic control problem, 
        where an agent tries to balance a pole on a cart. The agent's observation space consists of cart's 
        position, velocity, angle of the pole and its angular velocity. The agent can move the cart left or right.
        This problem is a test of the implementation's ability to solve a basic real-world problem, but contrary 
        to the xor problem, its model is subjected to multiple sequential runs,
        with its fitness being judged based on its cumulative performance across these runs. This means, that
        recurrent connections can potentially be used.
        \subsubsection{Acrobot}
        Acrobot is a similar problem to the cartpole, with a slightly bigger state space. The problem consists
        of a two-link system, one of its ends fixed to a certain position. All of the joints are allowed to rotate,
        abd the agent is given control over an actuated joint between the two links. Its goal is to swing the
        end of the free link over given height (while starting from a hanging position). 
        THis problem also is evaluated as a single evaluation session, being subjected to multiple sequential runs,
        with its fitness being judged based on its cumulative performance across these runs.
        \subsubsection{BipedalWalker}
        BipedalWalker is the hardest problem used in this work. It is a problem taken from Farama Foundation's
        Gymnasium library. The agent tries to control a bipedal robot and make it walk to the end of the map, while
        making its movements energy-efficient. The observation space is significantly bigger than in previous problems,
        consisting of 24 values. The action space somewhat similar, but due to the problem specifics, requiring a 
        much more precise control. 
    \subsection{Implementation details}
        \subsubsection{Core algorithm}
        The core algorithm if NEAT is implemented in ~400 lines of Python in model.py
        The base for a single agent is a class called Genome. It contains a list of nodes and a list of connections for this particular 
        specimen, as well as a fitness value. Nodes and Connections have their own respectively named classes, both of which are accessing
        their Genomes' config object to get information about their ids and innovation numbers. Genome also controls the mutation process of itself,
        having separate methods for mutations of nodes, connections, and weights. Genome's config is a shared object between all genomes, and it is 
        given to it by a Population object during its initialization. Population is a class that controls the whole algorithm. It is responsible for
        maintaining a list of genomes, list of species, and maintaining a coherent state between all the mutations of genomes and changes in specie set.
        It is also responsible for the toplevel loop, that runs the algorithm for a given number of generations. The single loop looks like this:
        \begin{enumerate}
            \item Divide the population into Species
            \item Calculate fitness of each individual genome
            \item Check for exit condition
            \item Calculate adjusted fitness of each individual genome
            \item Calculate offspring count for each specie
            \item Create new generation by crossover and mutation of selected genomes and species
        \end{enumerate}
        The Specie class is responsible for dealing with the speciation process and calculating relevant statistics for each (such as adjusted fitness or 
        generation without improvement count). 
        \subsubsection{Configuration}
        Population object is initialized with a Config object, which is a main way to configure algorithm. It is a simple object with a set of fields
        and a couple of methods. Most of the fields have default values, and are not required to be set manually by the user. The few fields that are
        mandatory are: 
        \begin{itemize}
            \item population\_size - the number of genomes in the population
            \item input\_size - the number of input nodes
            \item output\_size - the number of output nodes
            \item fitness\_function - a function that takes a genome and returns its fitness
        \end{itemize}
        TODO - tabelka z domyslnymi wartosciami konfiguracji
        TODO - opis pola "meta" (disjoint fields for probolems, so that they cannot overwrite each other's fields, and so that they cannot overwrite other important fields)
        \subsubsection{Problem-specific adjustments}
        TODO - steepened activation function for xor,
        TODO - different activation func for -1 - 1
        TODO - recurrent enabled disabled
        TODO - cross specie reproduction
        TODO - using gymnasium environment 
        TODO - adjusting fitness value from bipedal walker
        thoink of more
\section{Experimental results}
[here write about results of experiments]
\section{Conclusions}
[here write conclusions about results of experiments]

\section{Future work}
\section{Bibliography}

\end{document}
